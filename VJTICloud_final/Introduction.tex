\chapter{Introduction}
\makeatletter
%describe the problem 
% -describe diff types of problem, that were mentioned in papers and how were they solved (prevention methods)
% Why theres need to solve your problem
% How to solve the problem
% What are the improvements by your solution

In a world that sees new technological trends bloom and fade on almost a daily basis, one new trend promises more longevity. This trend is called cloud computing, and it will change the way you use your computer and the Internet\cite{miller}.

\par Cloud computing has introduced both simplicity and complexity in IT departments. The cloud model promises decreased costs, on-demand applications, high levels of service and increased efficiency, all in an easy-to-manage format. However, cloud computing can fall short in several areas, making it very important for enterprises to choose the cloud format that’s right for them.

\par For some organizations, a simple public cloud model involving outsourced operations to a cloud provider, usually for a fixed monthly fee is the answer. But the public cloud also comes with the potential hazards such as security and data protection issues, and it can also create roadblocks to scalability. Not all public cloud providers have solved for these problems. Companies can quickly find themselves at the mercy of third-party cloud providers lacking consistency for in service levels. The public cloud also assumes a company will store its critical data with a third-party provider, quickly becoming a roadblock for many organizations, particularly in health care and financial services. 

\par The hybrid cloud a mix of the public cloud and an internal private cloud managed in-house can offer greater scalability and increased control over the computing environment. But it can also bring challenges. Integrating elements of public and private clouds can be difficult and surprisingly expensive. Plus, with a third-party provider still in play, questions about security, data protection and management can still persist. 

\par For number of organizations, the private cloud a cloud-like model managed entirely in-house is proving to be the best option. Investing in a private cloud gives an enterprise control over its data and provides a higher level of security, while also enabling IT to manage its own service levels and scale applications as needed. Essentially, the private cloud leverages existing IT infrastructure, takes facets of the cloud and maps them into the enterprise. The result is a mix of cost savings, efficiency and control that brings the security of classic IT models with the efficiency of the cloud. 

\section{Problem Formulation}
 \par
 %\@title
 \textbf{\textquotedblleft \@title"}
 
 \par Main objective of this project is to build cloud for VJTI with core cloud computing services such as compute, storage and network. Currently We have Virtualization environment which is very good but has some limitaions like centralized control is required, manual deployment, cannot be made fully automated etc. Also every year there are at least 10 to 15 projects related to cloud are completed in college but they don't get cloud environment to use they have to depend on public clouds, so lets build the fully functional cloud. 

\section{Objective and Scope Work}

The main objective of the research presented here is to build private cloud infrastructure and find out the use cases for educational institute as well as provide guidelines for production level cloud.  

\section{Significant Contribution}

\section{Thesis Organization}

\makeatother