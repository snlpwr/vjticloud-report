\chapter{Introduction}
\makeatletter
%describe the problem 
% -describe diff types of problem, that were mentioned in papers and how were they solved (prevention methods)
% Why theres need to solve your problem
% How to solve the problem
% What are the improvements by your solution

In a world that sees new technological trends bloom and fade on almost a daily basis, one new trend promises more longevity. This trend is called cloud computing, and it will change the way you use your computer and the Internet\cite{miller}.

\par Cloud computing has introduced both simplicity and complexity in IT departments. The cloud model promises decreased costs, on-demand applications, high levels of service and increased efficiency, all in an easy-to-manage format. However, cloud computing can fall short in several areas, making it very important for enterprises to choose the cloud format that’s right for them.

\par For some organizations, a simple public cloud model involving outsourced operations to a cloud provider, usually for a fixed monthly fee is the answer. But the public cloud also comes with the potential hazards such as security and data protection issues, and it can also create roadblocks to scalability. Not all public cloud providers have solved for these problems. Companies can quickly find themselves at the mercy of third-party cloud providers lacking consistency for in service levels. The public cloud also assumes a company will store its critical data with a third-party provider, quickly becoming a roadblock for many organizations, particularly in health care and financial services. 

\par The hybrid cloud a mix of the public cloud and an internal private cloud managed in-house can offer greater scalability and increased control over the computing environment. But it can also bring challenges. Integrating elements of public and private clouds can be difficult and surprisingly expensive. Plus, with a third-party provider still in play, questions about security, data protection and management can still persist. 

\par For number of organizations, the private cloud a cloud-like model managed entirely in-house is proving to be the best option. Investing in a private cloud gives an enterprise control over its data and provides a higher level of security, while also enabling IT to manage its own service levels and scale applications as needed. Essentially, the private cloud leverages existing IT infrastructure, takes facets of the cloud and maps them into the enterprise. The result is a mix of cost savings, efficiency and control that brings the security of classic IT models with the efficiency of the cloud.

\section{Problem Formulation}
 \par
 %\@title
 \textbf{\textquotedblleft \@title"}
 
 \par VJTI is an engineering institute and research center and it has its own data center manged by using Proxmox Virtualization Environment. Though Proxmox provides stability and control over all resources of institute, it is not easy to manage all the things in peak hours, some system needs to keep shut off. Well this is technical issue lets talk about academics, every year there are at least three subjects related to cloud computing or cloud computing are taught as well as 10 to 15 students do their project related to cloud but they are not able to use cloud functionality unless they have subscription some public cloud. So, to solve above problems we come up with solution of- VJTICloud.

\section{Objective and Scope Work}

The main objective of the research presented here is to build private cloud using Openstack. Which will act as pilot project in the decision making whether VJTI should go for Private cloud or not. And also to address the following problems in current infrastructure 

\begin{enumerate}
  \item User should be able to start and shutdown his own VM
  \item Faster VM provisioning
  \item Scalability of hardware resources
  \item Full Hardware utilization and power saving % Flavors and IPMI interface or manually shuting down compute nodes when not needed.
  \item Multitenancy % Projects in dashboard
  \item Perform faster migration and live migration % HA shared storage
\end{enumerate}


\section{Significant Contribution}

Though we know that cloud can do this and cloud can do that, it is not easy to have all the features in private cloud. We have chosen Openstack for building the cloud IaaS, it is currently market leader in open source cloud software or cloud operating system. Openstack consist of interrelated components that control hardware pools of processing, storage and network resources\cite{wiki}. Openstack has following core projects / services
\begin{enumerate}
    \item Keystone / Identity service
    \item Nova / Compute service
    \item Glance / Image service
    \item Neutron / Network service
\end{enumerate}

and some optional projects / services:
\begin{enumerate}
    \item Horizon / Dashboard
    \item Cinder / Block storage service
    \item Swift / Object storage service
    \item Heat / Orchestration service
    \item Ceilometer / Telemetry service
    \item Sahara / Elastic Map Reduce service
    \item Trove / Database service
\end{enumerate}

and even if you install these services, all features of these services are not available by default you have to configure them. Following is a list of additional services and features needs to be added in VJTICloud.

\begin{enumerate}
  \item Block storage service
  \item Object storage service
  \item Dashboard service
  \item Orchestration service
  \item Firewall as a service
  \item Load balancer as a service
  \item VM resize 
  \item Live VM migration
  \item Make cloud available for institute
  \item GUI access of VM using browser
  \item Network configuration
  
\end{enumerate}

\section{Thesis Organization}

\makeatother
