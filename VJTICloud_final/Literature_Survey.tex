\chapter{Literature Survey}

% citation in heading
% method used
% prevention method
		% before 
		% after
		% use of solution	
% Research Gap
% solution of all papers 
% what problem taken care of 
% what are remaining 
% how they are affecting the system
\section{Cloud Computing}

\par Cloud computing is a software platform that provides computing services with the help of internet. It allows users to make use of the software and hardware which is managed by the parties at remote locations like online store, social networking, webmail, e-commerce applications etc. With the help of a cloud computing model it provides access to information and computer resources from anywhere in the world with network and internet connection. It provides shared pool of resources that includes data storage space, networking facilities, computing processing power of specialized and user applications\cite{deploy}.%

\subsection{Characteristics of Cloud computing}
There are five main characteristics of cloud computing service which differs it from traditional hosting service are
\par
\begin{enumerate}
\item On demand self service
    \par A user can self provision virtual computing resources such as server time, network and storage without requiring human interaction with the service provider\cite{wiki}.
\item Broad network access
    \par Users can access the cloud from anywhere, anytime from any device, it doesn't matter if it is thin or thick client. e.g. Mobile phones, tablets, laptops and workstation\cite{wiki}. 
\item Resource pooling
    \par The provider's computing resources are pooled to serve multiple consumers using a multi-tenant model, with different physical and virtual resources dynamically assigned and reassigned according to consumer demand\cite{wiki}. 
\item Rapid elasticity
    \par Capabilities can be elastically provisioned and released, in some cases automatically, to scale rapidly outward and inward commensurate with demand. To the consumer, the capabilities available for provisioning often appear unlimited and can be appropriated in any quantity at any time\cite{wiki}.
\item Measured service (Pay as you go)
    \par Cloud systems automatically control and optimize resource use by leveraging a metering capability at some level of abstraction appropriate to the type of service (e.g., storage, processing, bandwidth, and active user accounts). Resource usage can be monitored, controlled, and reported, providing transparency for both the provider and consumer of the utilized service\cite{wiki}.
\end{enumerate}


\subsection{Categories of Cloud Computing}
Cloud computing has three categories under which they can be deployed: private, public or Hybrid.
\par
\begin{enumerate}

\item Public cloud  
\par
In such cloud infrastructure the cloud service provider charges for their services from the organization. Like Amazon (AWS), Microsoft and Google, etc. they own and operate the infrastructure of the cloud and offer access to the services in the cloud with the help of Internet. Amazon Web Services (AWS) are the largest public cloud provider.

\item Private cloud
\par
In such cloud infrastructure the cloud computing platform is operated solely for a specific organization under the control of IT department and is managed by the organization behind the organizational firewall. Private cloud offers the same features as public cloud and eliminates the issues related to control, data, security etc.


\item Hybrid cloud
\par
Is a cloud infrastructure which combines both private and public cloud infrastructures together. They're entities remained as separate but offering the benefits of both the deployment environments.
\end{enumerate}

\subsection{Cloud Computing Models}
Cloud computing has three service models
\par
\begin{enumerate}
    \item Software as a Service (SaaS)
    \par
        In this type of service model, the cloud provider will provide the platform for the users to use their applications remotely like mail services , e-commerce applications etc. the users use this service will have the common interface which helps the companies not to pay for extra charges for the licenses\cite{deploy}.%
    \item Platform as a Service (PaaS)
    \par
        In this type of service model, the cloud provider provides and manages the platform (hardware, server operating system, databases, storage, architecture, networks and Virtualization) but the client has to develop its application as per the need using these services provided by the cloud provider. The client has to manage the applications and the cloud service provider manages everything else. Like google app engine\cite{deploy}.%\cite{deploy}
    \item Infrastructure as a Service (IaaS)
    \par
        In this service model, the cloud service provider provides and manages the infrastructure (Virtualization, servers, networking and storage). This helps to avoid spending on hardware and human resources and which reduces the return of investment risk. The client can execute and manages virtual machines where it can use its applications, data, operating system, middleware and runtime like Amazon EC2, Rackspace, etc\cite{deploy}.%\cite{deploy}
\end{enumerate}
