\chapter{Literature Survey}
    \section{Related work}
    \par Lakshmi D Kurup et. al \cite{lakshmi} introduce us to service models of cloud and comparative study of open source cloud computing platforms. The paper describes popular cloud platforms for deploying Infrastructure as a Service such as Eucalyptus, OpenStack and Nimbus. For each framework, they illustrated its origin, architecture and design as well as provided a feature comparison. In their conclusion they have mentioned that OpenStack is the best choice for building public cloud.
    
    Aaron Paradowski, Lu Liu, and Bo Yuan\cite{cloudstack} has examined the performance variations between OpenStack and CloudStack. Their benchmarking criteria clearly shows that performance of OpenStack is better than CloudStack.
    
    Xiaolong Wen et al.\cite{opennebula} have shown the differnce between OpenStack and OpenNebula from provenance, architecture, hypervisors and security. Also given an idea about implementation of the two.

    \section{Cloud Computing}
    
    \par Cloud computing is a software platform that provides computing services with the help of internet. It allows users to make use of the software and hardware which is managed by the parties at remote locations like online store, social networking, webmail, e-commerce applications etc. With the help of a cloud computing model it provides access to information and computer resources from anywhere in the world with network and internet connection. It provides shared pool of resources that includes data storage space, networking facilities, computing processing power of specialized and user applications\cite{deploy}.%
    
    \subsection{Characteristics}
    There are five essential characteristics of cloud computing service which differs it from traditional hosting service are
    
    \par
    \begin{enumerate}
    \item \textbf{On demand self service}
        \par A user can self provision virtual computing resources such as server time, network and storage without requiring human interaction with the service provider\cite{wiki}.
    \item \textbf{Broad network access}
        \par Users can access the cloud from anywhere, anytime from any device, it doesn't matter if it is thin or thick client. e.g. Mobile phones, tablets, laptops and workstation\cite{wiki}. 
    \item \textbf{Resource pooling}
        \par The provider's computing resources are pooled to serve multiple consumers using a multi-tenant model, with different physical and virtual resources dynamically assigned and reassigned according to consumer demand\cite{wiki}. 
    \item \textbf{Rapid elasticity}
        \par Capabilities can be elastically provisioned and released, in some cases automatically, to scale rapidly outward and inward commensurate with demand. To the consumer, the capabilities available for provisioning often appear unlimited and can be appropriated in any quantity at any time\cite{wiki}.
    \item \textbf{Measured service}
        \par Cloud systems automatically control and optimize resource use by leveraging a metering capability at some level of abstraction appropriate to the type of service (e.g., storage, processing, bandwidth, and active user accounts). Resource usage can be monitored, controlled, and reported, providing transparency for both the provider and consumer of the utilized service\cite{wiki}.
    \end{enumerate}
    
    \par Following are the other characteristics of cloud computing
    
    \begin{enumerate}
        \item \textbf{Device and location independence} enable users to access systems using a web browser regardless of their location or what device they use (e.g., PC, mobile phone). As infrastructure is off-site (typically provided by a third-party) and accessed via the Internet, users can connect from anywhere.
        
        \item \textbf{Maintenance} of cloud computing applications is easier, because they do not need to be installed on each user's computer and can be accessed from different places.
        
        \item \textbf{Reliability} improves with the use of multiple redundant sites, which makes well-designed cloud computing suitable for business continuity and disaster recovery.
        
        \item \textbf{Scalability and elasticity} via dynamic ("on-demand") provisioning of resources on a fine-grained, self-service basis in near real-time (Note, the VM startup time varies by VM type, location, OS and cloud providers), without users having to engineer for peak loads. This gives the ability to scale up when the usage need increases or down if resources are not being used.
        
        \item \textbf{Security} can improve due to centralization of data, increased security-focused resources, etc., but concerns can persist about loss of control over certain sensitive data, and the lack of security for stored kernels. Security is often as good as or better than other traditional systems, in part because providers are able to devote resources to solving security issues that many customers cannot afford to tackle.
        
        \item \textbf{Multitenancy} enables sharing of resources and costs across a large pool of users.
        
    \end{enumerate}
    
    \subsection{Deployment Models}
    Cloud computing has three categories under which they can be deployed: private, public or Hybrid.
    
    \begin{figure}[h]
        \centering
        \includegraphics[width=13cm,height=9cm]{images/types_of_cloud.png}
        \caption{Deployment models of cloud}
    \end{figure}
    
    \par
    \begin{enumerate}
    
    \item \textbf{Public cloud}  
    \par
    In such cloud infrastructure the cloud service provider charges for their services from the organization. Like Amazon (AWS), Microsoft and Google, etc. they own and operate the infrastructure of the cloud and offer access to the services in the cloud with the help of Internet. Amazon Web Services (AWS) are the largest public cloud provider\cite{deploy}.
    
    \item \textbf{Private cloud}
    \par
    In such cloud infrastructure the cloud computing platform is operated solely for a specific organization under the control of IT department and is managed by the organization behind the organizational firewall. Private cloud offers the same features as public cloud and eliminates the issues related to control, data, security etc\cite{deploy}.
    
    \item \textbf{Hybrid cloud}
    \par
    Is a cloud infrastructure which combines both private and public cloud infrastructures together. They're entities remained as separate but offering the benefits of both the deployment environments\cite{deploy}.
    \end{enumerate}
    \clearpage
    \subsection{Service Models}
    Cloud computing has three service models
    
    \begin{figure}[h]
        \centering
        \includegraphics[width=14cm,height=10cm]{images/service_models2.png}
        \caption{Service models of cloud}
    \end{figure}
    
    \par
    \begin{enumerate}
        \item \textbf{Software as a Service (SaaS)\cite{deploy}}
        \par
            In this type of service model, the cloud provider will provide the platform for the users to use their applications remotely like mail services , e-commerce applications etc. the users use this service will have the common interface which helps the companies not to pay for extra charges for the licenses.%
        
        \item \textbf{Platform as a Service (PaaS)\cite{deploy}}
        \par
            In this type of service model, the cloud provider provides and manages the platform (hardware, server operating system, databases, storage, architecture, networks and Virtualization) but the client has to develop its application as per the need using these services provided by the cloud provider. The client has to manage the applications and the cloud service provider manages everything else. Like google app engine.%\cite{deploy}
        
        \item \textbf{Infrastructure as a Service (IaaS)\cite{deploy}}
        \par
            In this service model, the cloud service provider provides and manages the infrastructure (Virtualization, servers, networking and storage). This helps to avoid spending on hardware and human resources and which reduces the return of investment risk. The client can execute and manages virtual machines where it can use its applications, data, operating system, middleware and runtime like Amazon EC2, Rackspace, etc.
    \end{enumerate}
    
    \subsection{Benefits of cloud computing}
    \begin{enumerate}
    \item Flexibility
    \par Cloud-based services are ideal for businesses with growing or fluctuating bandwidth demands. If your needs increase it’s easy to scale up your cloud capacity, drawing on the service’s remote servers. Likewise, if you need to scale down again, the flexibility is baked into the service\cite{ben}. 
    
    \item Disaster recovery
    \par Businesses of all sizes should be investing in robust disaster recovery, but for smaller businesses that lack the required cash and expertise, this is often more an ideal than the reality. Cloud is now helping more organisations buck that trend\cite{ben}.
    
    \item Automatic software updates
    \par The beauty of cloud computing is that the servers are off-premise, out of sight and out of your hair. Suppliers take care of them for you and roll out regular software updates – including security updates – so you don’t have to worry about wasting time maintaining the system yourself. Leaving you free to focus on the things that matter, like growing your business\cite{ben}.
    
    \item Capital-expenditure Free
    \par Cloud computing cuts out the high cost of hardware. You simply pay as you go and enjoy a subscription-based model that’s kind to your cash flow. Add to that the ease of setup and management and suddenly your scary, hairy IT project looks at lot friendlier\cite{ben}. 
    
    \item Increased collaboration
    \par When your teams can access, edit and share documents anytime, from anywhere, they’re able to do more together, and do it better. Cloud-based workflow and file sharing apps help them make updates in real time and gives them full visibility of their collaborations\cite{ben}.
    
    \item Work from anywhere
    \par With cloud computing, if you’ve got an internet connection you can be at work. And with most serious cloud services offering mobile apps, you’re not restricted by which device you’ve got to hand\cite{ben}.
    
    \item Document control
    \par The more employees and partners collaborate on documents, the greater the need for watertight document control. Before the cloud, workers had to send files back and forth as email attachments to be worked on by one user at a time. Sooner or later – usually sooner – you end up with a mess of conflicting file content, formats and titles\cite{ben}.
    
    \item Security
    \par Lost laptops are a billion dollar business problem. And potentially greater than the loss of an expensive piece of kit is the loss of the sensitive data inside it. Cloud computing gives you greater security when this happens. Because your data is stored in the cloud, you can access it no matter what happens to your machine. And you can even remotely wipe data from lost laptops so it doesn’t get into the wrong hands\cite{ben}.
    
    \item Competitiveness
    \par Wish there was a simple step you could take to become more competitive? Moving to the cloud gives access to enterprise-class technology, for everyone. It also allows smaller businesses to act faster than big, established competitors. Pay-as-you-go service and cloud business applications mean small outfits can run with the big boys, and disrupt the market, while remaining lean and nimble\cite{ben}.
    
    \item Environmentally friendly
    \par While the above points spell out the benefits of cloud computing for your business, moving to the cloud isn’t an entirely selfish act. The environment gets a little love too. When your cloud needs fluctuate, your server capacity scales up and down to fit. So you only use the energy you need and you don’t leave oversized carbon footprints\cite{ben}.
    
    \end{enumerate}
    
    \section{Open source cloud platforms}
    
    Open source provides flexibility to the users to choose the product and even provide freedom to change the source code for own user's need. This brings openness and makes the product more effective for further future uses. This ability of freedom and openness is encouraging more and more programmers. Who are migrating towards to work on open source cloud packages as don't have to pay, look over proprietary issues. An open source cloud is growing and becoming more effective for the IT Industries and organizations who wanted to use the cloud facilities for hosting and other services. Not only OpenStack is growing in other open source cloud are also growing like cloudstack, openNebula, Eucalyptus etc.
    
    \par Following are the top open source cloud computing platforms\cite{top} for building Infrastructure as a Service
    
    % http://www.tomsitpro.com/articles/open-source-cloud-computing-software,2-754-4.html
        \subsection{Apache CloudStack}
        \par Despite rumors to the contrary, Java continues to prove central to many major cloud applications. At the heart of Apache CloudStack is a host of functions written in Java including user management, multi-tenancy and account separation, network, compute and storage resource accounting, web-based management console, native API and Amazon S3/EC2 compatible API, and primary/secondary storage support. Apache CloudStack works with hosts on XenServer/XCP, KVM, Hyper-V and VMware. Used to deploy and manage large networks of virtual systems, Apache CloudStack has been chosen by many providers deploying private, public, and hybrid cloud solutions to customers. Additional features include high availability, a scalable infrastructure as a service cloud computing platform, and a significant community of users and developers who keep the technology and feature improvements moving forward.
        
        \subsection{Eucalyptus}
        
        \par Though currently only available on CentOS and Red Hat Enterprise Linux, Eucalyptus is already getting notice as a complete IaaS solution. Comprised of a Cloud Controller (CLC), Walrus (persistent data storage), Cluster Controller (CC), Storage Controller (SC), Node Controller (NC), and an optional VMware Broker (VB), Eucalyptus is a full-featured product. Each component is a stand-alone web service (excluding VB), with the aim of allowing Eucalyptus to provide an API for each service (language-agnostic). This Linux-based system allows users to implement private and hybrid clouds within existing infrastructure with an industry-standard, modular framework. In particular, Eucalyptus provides a virtual network overlay isolating various traffic, allowing multiple clusters to be transparent on the same Local Area Network (LAN) while maintaining data integrity. Additionally, Eucalyptus is API compatable with Amazon’s EC2, S3, IAM, ELB, Auto Scaling, and CloudWatch services, ideal for hybrid cloud implementation options.
    
    
        \subsection{OpenNebula}
        
        \par A combination of functional project and research, OpenNebula purports to be the next step in the evolution of data center virtualization. From a research perspective, the project seeks to develop advanced and adaptable virtualization data centers and enterprise clouds. Through collaboration with other open source projects and researchers in cloud computing, OpenNebula hopes to achieve stability and quality of cloud computing software, as well. The project's core values include process and technology openness, excellence across all project lifecycles, and innovation in cloud development. Regarding their actual functional product, key features of this are currently reported to be an intuitive self-service portal, automated service management catalog, administration and super user interfaces, appliance marketplace, performance and capacity management, high availability, business continuity, virtual infrastructure management, enterprise-level security, third-party tool integration and excellent product support and SLA-based commercial support directly from the developers.
    
        \subsection{OpenStack}
        
        \par Of all the IaaS offerings, OpenStack is one of only a couple that appear in multiple product areas of cloud computing architecture. A global project, OpenStack was founded by Rackspace and NASA, who produced a massively scalable cloud operating system, freely available under the Apache 2.0 license. OpenStack has no proprietary hardware or software requirements, and is designed to operate within both fully virtual and bare metal systems. Multiple hypervisors are supported, including KVM and XenServer, as well as container technology, including LXC. OpenStack is used anywhere from service providers deploying IaaS to its customers, to enterprise IT departments providing private cloud services to project teams and departments. OpenStack works with Hadoop for big data needs, scales vertically and horizontally to meet diverse computing needs, and offers high-performance computing (HPC) for intensive workloads. Key features include VM image caching, role based access control, VM image management, LAN management, VNC proxy via web browser, floating IP addresses, and much more.